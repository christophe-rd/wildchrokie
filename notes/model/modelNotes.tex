\documentclass[a4paper,12pt]{article}
\usepackage{amsmath, amssymb, hyperref, listings}
\usepackage{xcolor}
\usepackage{geometry}
\usepackage{textgreek}

\geometry{margin=1in}
\setlength\parindent{0pt}

\title{CoringTreespotters Model notes}
\author{Christophe}
\date{\today}


\begin{document}

\maketitle

\section*{15 April 2025}
Quick notes after meeting with Victor, where we discussed the preliminary steps for building a model to address how juvenile trees respond to climate change. I’d like to frame this in terms of the recruitment capacity of tree saplings—looking at how their growth varies across years and how it’s influenced by temperature and the length of the growing season (or perhaps not the length specifically... more on that below).

\section{Come up with a model}

\subsection*{What are my predictors?}
We discussed which of the following would be the most relevant variable that relates to growth:
\begin{enumerate}
	\item \textbf{Growing Degree Days (GDD)} which would be calculated between the budburst and budset date of each individual. 
	\item \textbf{Growing Season Length (GSL)} which could be calculated by:
	\begin{enumerate}
		\item Substracting budset DOY by budburst DOY or 
		\item By taking the number of days of each year when the mean (or max/min) was above 5$^{\circ}$C (or maybe the nb of consecutive days when the temperature was above 5 or something like that).
	\end{enumerate}
\end{enumerate}

\par
\textbf{Preliminary model} \\ 
\begin{align}	
	\log (w_{i,t}) \sim \text{normal}(\text{X}, \sigma) \\
	\alpha + \alpha_{sp[i]} + \beta_{sp[i]} X_{i,t} + \cdots
\end{align} 
What kind of pooling? Partial pooling since they are all within the same family?\\

\section{Simulate data}
I arbitrarily decided to sum up the GDD between DOY 100 and 300 for data simulation. \\
yhat: combine the effects that I generated from the grand mean, the year effect (other than variation of GDD), individual effect, species response to GDD and the overall error, which returns my yhat i.e. ring width in mm.

\section{Set your priors}

\section{Run model on empirical data}

\section{Perform retrodictive checks using the model fit to your empirical data}

\section{Mis}

\newpage
\textit{Below I will be stating the o bvious because I want to make sure I understand the notations and stuff. The steps will be notes and thoughts that I have when I code my model in R.}

\end{document}
