

\documentclass{article}
\usepackage[utf8]{inputenc}
\usepackage{authblk}
\usepackage{setspace}
\usepackage[margin=1.25in]{geometry}
\usepackage{graphicx}
\graphicspath{ {./figures/} }
\usepackage{subcaption}
\usepackage{amsmath}
\usepackage{lineno}
\linenumbers


%%%%%% Bibliography %%%%%%
% Replace "sample" in the \addbibresource line below with the name of your .bib file.
\usepackage[style=nejm, 
citestyle=numeric-comp,
sorting=none]{biblatex}
\addbibresource{sample.bib}

%%%%%% Title %%%%%%
% Full titles can be a maximum of 200 characters, including spaces. 
% Title Format: Use title case, capitalizing the first letter of each word, except for certain small words, such as articles and short prepositions
\title{WildchrokieXTreespotters draft manuscript}

%%%%%% Authors %%%%%%
% Authors should be listed in order of contribution to the paper, by full first name, then middle initial (if any), followed by last name and separated by commas.
% Please do not use initials for first names. If you use your middle name as a full name, use an initial for the first name and spell out your full middle name.
% Use a superscript asterisk (*) to identify the corresponding author and be sure to include that person’s e-mail address. Use symbols (in this order: †, ‡, §, ||, ¶, #, ††, ‡‡, etc.) for author notes, such as present addresses, “These authors contributed equally to this work” notations, and similar information.
% You can include group authors, but please include a list of the actual authors (the group members) in the Supplementary Materials.
\author[1*$\dag$]{Christophe Rouleau-Desrochers}


%%%%%% Affiliations %%%%%%
\affil[1]{UBC}
%%%%%% Date %%%%%%
% Date is optional
\date{June 10, 2024}

%%%%%% Spacing %%%%%%
% Use paragraph spacing of 1.5 or 2 (for double spacing, use command \doublespacing)
\onehalfspacing

\begin{document}

\maketitle

%%%%%% Abstract %%%%%%
\begin{abstract}
\end{abstract}
%%%%%% Main Text %%%%%%

\section{Introduction}

%%%%%%
\section{Materials and Methods}
%%%%%
\subsection{Wildchrokie}
% here is a copy paste from the wildhell manuscript that's already written
\subsubsection{The common garden at Weld Hill}
In 2014-2015, we collected seeds of 18 species woody plants from four field sites in eastern Northern America spanning a Y degree latitudinal gradient. The four field sites included Harvard Forest (42.55$^{\circ}$N, 72.20$^{\circ}$W), the White mountains (44.11$^{\circ}$N, 52.14$^{\circ}$W), Second College Grant of Dartmouth College (44.79$^{\circ}$N, 50.66$^{\circ}$W), and St. Hippolyte, CN (45.98$^{\circ}$N, 74.01$^{\circ}$W). We transported all seed back the the Weld Hill Research Building at the Arnold Arboretum in Boston MA (lat long) where we germinated seeds following standard germination protocols, and grew them to seedling stages for a year or so in the research green house. In Spring of 2017 we planted them out to establish the common garden at Weld hill. Plantings were randomized between 16 plot blocks. Individuals that were too small to survive outside were maintained in the growth facilities for an additional year and out planted in the early spring of 2018 (i think - yeah that's right!). Plots were divided between tree plots which included species x,y,z and shrub plots which included species a,b,c,d and shade cloth (Table \ref{listSp}). Plots were regularly weeded and watered throughout the duration of the study.

\subsection{Cookie collection at Weld Hill}
In the spring and the fall of 2023, we collected tree cookies for 5 species: Alnus incana, Betula allenghianis, Betula populifolia, Betula papyfera. These trees were left to dry for (**I don't know how long) at ambient temperature. Then they were sanded at progressive grit until a very fine grit of 1500. We scanned the cookies using Tina that get us a resolution of 6350 DPI. We added 3 guidelines per cookie at 180 degrees of each other to ensure consistency in the orientation of the measurements. Then, we measured ring width using the Fiji is just imageJ, by manually drawing a line on each ring. Each ring was measured 3 times following the guidelines and set the lines at 90 degrees of the ring boundary. 

\subsection {Treespotters phenological measurements}

\subsection {Treespotters coring}
In April 2025, we cored 50 trees that were monitored for 10 years by a citizen science project called the Treespotters. The cores were placed in straws and left to dry for 45 days at ambient condition. Then the cores were sanded and scanned and blabla. 

\subsection{Experimental Design}
%\includegraphics[]{../experimental_design/Fuelinex_Design_V4.jpg}

\subsection{Statistical Analysis}
%%%%%%
`section{Results}
%%%%%%
\section{Discussion}
%%%%%%

\printbibliography

\end{document}
