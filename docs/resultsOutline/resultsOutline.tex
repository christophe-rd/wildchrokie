\documentclass{article}
\usepackage{hyperref}
\usepackage[margin=1in]{geometry}
\usepackage{Sweave}
\begin{document}


\title{Results outline}
\author{Christophe Rouleau-Desrochers}
\date{\today}
\maketitle
\begin{Schunk}
\begin{Soutput}
SCRIPT STARTED
\end{Soutput}
\end{Schunk}
% <><><><><><><><><><><><><><><><><><><><><><><><><><><><><><><><><><><><><><><>
% Main results
% <><><><><><><><><><><><><><><><><><><><><><><><><><><><><><><><><><><><><><><>
\section*{Main results}

\subsection*{Wildchrokie}
\begin{enumerate}
  \item ALNINC, BETALL, BETPAP grew more with more growing degree days, while BETPOP did not respond to growing degree days (Figure \ref{fig:wcgrowthSlopes}).
  \item BETPAP had the highest growth rate and BETALL the lowest (Figure \ref{fig:wcmuplotbspp})
  \item For the provenance we did not detect a growth difference between WM, SH GR and only HF grew less (Figure \ref{fig:wcmuplotbsppasite})
\end{enumerate}

I want to invoke y: 5.977, now if I want to invoke the difference between bspp 1 and bspp 2: -0.273


% Figure Growth X GDD
\begin{figure}[!ht]
\includegraphics[width=1\textwidth]{../../analyses/figures/empiricalData/growthModelSlopesperSppFacet.jpeg}
\caption{Ring width responses to growing degree days for the primary growing season. The shaded areas represent the 50\% confidence intervals. The dots represent the empirical data}
\label{fig:wcgrowthSlopes}
\end{figure}

% Figure mu plot bspp
\begin{figure}[!ht]
\includegraphics[width=0.5\textwidth]{../../analyses/figures/empiricalData/bspp_mean_plot.jpeg}
\caption{Slope values across the different species}
\label{fig:wcmuplotbspp}
\end{figure}

% Figure provenances
\begin{figure}[h!]
\includegraphics[width=0.5\textwidth]{../../analyses/figures/empiricalData/asite_mean_plot.jpeg}
\caption{Site intercept values}
\label{fig:wcmuplotbsppasite}
\end{figure}
%-------------------------------------------------------------------------------
%-------------------------------------------------------------------------------
\subsection*{coringTreespotters}
\begin{enumerate}
  \item Both maples, and both birches responded to varying growing degree days during the growing season. Maples and River birch grew more wheras Yellow birch grew less.(Figure \ref{fig:growthSlopes})
  \item
  \item
  \item 
\end{enumerate}


% Figure Growth X GDD
\begin{figure}[!ht]
\includegraphics[width=1\textwidth]{../../../coringtreespotters/analyses/figures/empiricalData/growthModelSlopesperSppFacet.jpeg}
\caption{Ring width responses to growing degree days for the primary growing season. The shaded areas represent the 50\% confidence intervals. The dots represent the empirical data}
\label{fig:tsgrowthSlopes}
\end{figure}

% Figure mu plot bspp
\begin{figure}[!ht]
\includegraphics[width=0.5\textwidth]{../../analyses/figures/empiricalData/bspp_mean_plot.jpeg}
\caption{Slope values across the different species}
\label{fig:tsmuplotbspp}
\end{figure}

% Figure provenances
\begin{figure}[h!]
\includegraphics[width=0.5\textwidth]{../../analyses/figures/empiricalData/asite_mean_plot.jpeg}
\caption{Site intercept values}
\label{fig:tsmuplotbsppasite}
\end{figure}

% <><><><><><><><><><><><><><><><><><><><><><><><><><><><><><><><><><><><><><><>
% Supplemental results
% <><><><><><><><><><><><><><><><><><><><><><><><><><><><><><><><><><><><><><><>
\section*{Supplemental results}

\subsection*{Wildchrokie}

\end{document}
